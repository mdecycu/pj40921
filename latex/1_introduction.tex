\chapter{前言}
\renewcommand{\baselinestretch}{10.0} %設定行距
\pagenumbering{arabic} %設定頁號阿拉伯數字
\setcounter{page}{1}  %設定頁數
\fontsize{14pt}{2.5pt}\sectionef
\section{研究動機}
高中職到大學現在所教的繪圖軟體都是老師在業界中所挑選出來的,不外乎最主要的軟體就是 Soidworks ,身為業界中中小企業最受歡迎的程式,只要是有畫 3D 繪圖的人都一定知道的,但在專題老師的介紹下,我們得知了不輸於 Solidworks 的繪圖軟體 Solid Edge 以及同公司旗下的限元素分析軟體 Femap 。 Solidworks 在台灣機械領域已經有個不可動搖的地位存在了,但我們必須讓這些不亞於 SW 的優良軟體浮出水面,讓更多的人知道不是只要學會 Solidworks 就行了!\\

我們將Solid Edge比Solidworks更有優勢的部分列出了以下幾點:\\
\begin{itemize}
\item 使用授權:\\
學生能在官網申請後,得到1000天的使用授權,且能用Solid Edge所有的軟體,無須因授權問題而使用非正版的軟體\\
\item 可攜版本\\
軟體不須透過網路取得權限,能直接將Solid Edge安裝在隨身硬碟上隨處使用\\
\item 分析軟體\\
Solid Edge內建Famap有限元素分析軟體,不用經過轉檔,建模完能直接分析\\
\item 獨有技術\\
同步建模與工程參照的讓建模時間大幅縮短,大幅減省時間與人力成本\\

\section{Solid Edge 簡介}
Solid Edge 是一個完整的混合式 2D/3D CAD 系統,從零件、組裝、工程圖等研發工作流程一次完成。並搭載 Siemens 獨家 同步建模技術 ,能夠加速設計、更快的設計變更、提高匯入資料重用效率。

同步建模技術結合了速度和靈活性,直接建模,控制精確的尺寸驅動設計(功能和同步解決相關的參數)。參數的關係,可直接應用於固體功能,而不必依賴於二維草圖幾何關係和共同參數自動應用。該建模過程是表示要爭取一定的 CAD 設計活動高達100倍的速度。

不像其他直接建模系統,它不是典型的帶動歷史的建模方法,而不是提供參數化尺寸驅動的建模幾何通過同步,參數和使用規則的決策引擎,讓用戶使用不可預測的變化。這個對象驅動的編輯模式是被稱為對象操作界面,它強調一個用戶界面,提供直接操縱的對象( DMUI )。

2007年, SIEMENS (西門子股份有限公司)工業自動化部門收購了 UGS 公司,後將 UGS 的公司更名為 Siemens PLM Software 。 \\
\section{Femap 簡介}
 Femap 是一種嵌入在 Siemens PLM Software 的分析軟體,它提供了一個循序漸進的方法,以單個組件的分析建立 CAD 軟件系統的 Solid Edge CAD 解決方案,其成本低廉的高性能 FEA 建模,也適合工程師使用。 Femap 公認為全球領先獨立的 CAE 前後處理器,可用於進階工程有限元素分析,可以快速、輕鬆地確定過程中的組件是否滿足強度要求。 \\



\renewcommand{\baselinestretch}{0.5} %設定行距
